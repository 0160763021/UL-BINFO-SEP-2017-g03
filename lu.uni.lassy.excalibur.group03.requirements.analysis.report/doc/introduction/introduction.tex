% Last Modification:
% @author AUTHOR_NAME
% @date TODAY_DATE

\chapter{Introduction}
\label{chap:introduction}

\section{Overview}

VMMS is a simple system dedicated to companies which want to create virtual
machines for self-use or to make available to third parts. This allows to
whoever using these virtual machines to work on cloud and save themselves money
or space from buying physical hardware. The use of cloud is very effective
because you can specify every part of your virtual machine, modify it and delete
it as whished. The VMMS system has for objectif to support virtual machine
creation and effective control over it and safe use guarenteed by the super
administrator.

\section{Purpose and recipients of the document}

This document is an analysis document complying with the Messir methodology. Its intent is
to provide an example of a precise specification of the functional properties of
the VMMS.

The recipients of this document are:

� the VMMS�s buyer company (ABC): this document is used as a contractual
document jointly with any other document considered as useful (as requirement elicitation document, . . . )
in order to have a higher degree of precision in requirement description. It is also used as a basis
document for the iCrash system validation using specification based testing.

� the iCrash system development company (ADC) is expected to use this document as the basis
for development (mainly design, implementation, maintenance). It is also used for verification
and validation using test plans defined using the analysis models described in this document and
according to the Messir methodology.

\section{Application Domain}

The VMMS belongs to the Virtual Machine Management System's Domain. It is a
system dedicated to cloud professional companies and non professional client
users. It is an external service for client users of the application owner
company.

\section{Definitions, acronyms and abbreviations}

VMMS stands for Virtual Machine Management System.

\section{Document structure} 

The document structure is designed to be coherent with the Messir methodology. Section
2 provides a general description of the system purpose, its users, its environment and some general
non functional requirements. A more detailed description of the non functional requirements, if any,
are provided in section ??. The system operation triggered by events sent by the external actors
belonging to the environment are described in Section 3. The VMMS concepts used
to represent the any persistent or transient information is given in Section 4. The precise specification of the system
operations in term of system�s state changes, events sent together with the constraints on the allowed
sequences of system operations are described in Section 5.
