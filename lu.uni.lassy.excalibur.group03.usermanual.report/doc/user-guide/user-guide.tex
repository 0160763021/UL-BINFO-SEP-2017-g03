\chapter{Usage Guide}
\label{chap:usage_guide}

This section is aimed at describing the general use of the software, since it is
\textbf{deployed, configured} and \textbf{run}.

This software is used by actors. These actors rely on the software to perform a
set of business activities (called here procedures) aimed at reaching a
particular goal. 

These prodedures are splet in two groups:
\begin{itemize}
  \item \textbf{Multi-procedures:} which are procedures at \textbf{summary} or
  \textbf{user-goal} level involving several active or pro-active actors.
  Each of these procedures aims at illustrating intertwined
  business activities required to be performed by the involved actors
  to reach the expected goal. Each business activity between the system and an
  actor must correspond to a \textbf{system operation} instance given with actual parameter values.

  \item \textbf{Mono-procedures:} which are procedures at \textbf{summary} or
  \textbf{user-goal} level involving only one active or pro-active actor.
  Each of these procedures aims at illustrating the required business
  activities an actor has to perform to reach the expected goal. Each business
  activity between the system and the actor must correspond to a \textbf{system
  operation} instance given with actual parameter values.

\end{itemize}




Each process has to be documented using the following textual description
template \cite{armour01usecase} \textbf{BUT its content must be as low level as possible with actual values}:
\vspace{0.5cm}
\hrule
\begin{lyxlist}{PC1}
\small{
\item [\textbf{Procedure:}] ProcessMissionOne
\item [\textbf{Scope:}] Crisis Management System (\emph{CMS})
\item [\textbf{Primary Actor}:] Coordinator John
\item [\textbf{Secondary Actor(s)}:] FirstAidWorker Bob,\\
                  ExternalResourceSystem ERS
\item [\textbf{Goal:}] The intention of the Coordinator is to process mission
with ID equal to 1.
\item [\textbf{Level}:] User-goal level
\item [\textbf{Main~Success~Scenario}]:\\
1. \emph{John} instructs the \emph{CMS} to process the mission with ID equal to 12.031005\\
2. \emph{CMS} selects the internal worker \emph{Bob} to execute the mission 12.031005\\
3. \emph{CMS} instructs \emph{Bob} to behave as \emph{First Aid Worker (FAW)}\\
4. \emph{Bob} informs the \emph{CMS} of his arrival\\
5. \emph{Bob} informs the \emph{CMS} that he starts to execute the mission 12.031005\\
6. \emph{Bob} informs the \emph{CMS} that the mission 12.031005 outcome is ``Mission completed''


\item [\textbf{Extensions}]:\\
2.a None internal worker can execute the mission\\
\hspace*{0.5cm} 2.a.1 \emph{CMS} sends a request for an external resource to the \emph{ERS} actor instance\\
\hspace*{0.5cm} 2.a.2 \emph{ERS} informs \emph{CMS} that the request can be processed\\
\hspace*{0.5cm} 2.a.3 \emph{ERS} informs \emph{CMS} that \emph{Bob} can now be selected as first aid worker\\
\hspace*{0.5cm} \textbf{procedure continues at step 3}

}

\end{lyxlist}
\hrule
\vspace{0.5cm}




\Remark{Processes presentation}: processes should be introduced to the
reader in a pedagogical manner. Thus, simple and common processes should be presented before
than more complex and less utilised ones.

\Remark{Graphical User Interfaces (GUIs)}: include GUIs screenshots to show the
different stages of the process while its is performed by the actor(s).






\section{Multi-procedures}


\subsection{MyMultiProcedure1}
\ldots

\subsection{MyMultiProcedure2}
\ldots


\subsection{MyMultiProcedure3}
\ldots

\section{Mono-procedures}
Mono-procedures must be grouped by actors.


\subsection{SysAdmin}

\subsubsection{Creating a virtual machine using expert mode}

\hrule
\vspace{0.5cm}
\begin{lyxlist}{PC1}
\small{
\item [\textbf{Procedure:}] vmCreationProcess1
\item [\textbf{Scope:}] Virtual machine creation
\item [\textbf{Primary Actor}:] SysAdmin John
\item [\textbf{Secondary Actor(s)}:] /
\item [\textbf{Goal:}] The SysAdmin John should be able to create a virtual
machine which was created with his components that he choosed.
\item [\textbf{Level}:] User-goal level
\item [\textbf{Main~Success~Scenario}]:\\
1. \emph{John} must click on the button �VIRTUAL MACHINE� in the main Menu.
Fig 3.1\\
2. \emph{John} must click on the button with a plus named �CREATE VM�. Fig
3.2\\
3. \emph{John} must click on the button with a screwdriver called 'OWN
SETTINGS'.
Fig 3.3\\
4. \emph{John} must enter a name for the virtual machine and a description
(optional). Fig 3.4\\
5. \emph{John} must select a component. Fig 3.5\\
6. \emph{John} must click on the right arrows which allows him to
proceed to the next component and the left arrow allows him to go one page
back.\\
7. \emph{John} must repeat for every component e.g CPU, GPU, HDD, SSD,
SOFTWARE step 5 and 6. Fig 3.5 - 3.10\\
8. \emph{John} will be at the finish page, here John has now the option to
choose if he wants to create the selected VM, for this he must click on the
green check mark, and if he wants to abandon the configuration he has to
click on the red cross.
Fig 3.11\\






\item [\textbf{Extensions}]:\\
2.a John has to be logged in, in order to create a virtual machine.







\item [\textbf{GUI screenshot guide}]:\\

}

\end{lyxlist}
\hrule
\vspace{0.5cm}





\subsubsection{Creating a virtual machine using templates}

\hrule
\vspace{0.5cm}
\begin{lyxlist}{PC1}
\small{
\item [\textbf{Procedure:}] vmCreationProcess2
\item [\textbf{Scope:}] Virtual machine creation
\item [\textbf{Primary Actor}:] SysAdmin John 
\item [\textbf{Secondary Actor(s)}:] /
\item [\textbf{Goal:}] The SysAdmin John should be able to create a virtual
machine which was created using a template.
\item [\textbf{Level}:] User-goal level
\item [\textbf{Main~Success~Scenario}]:\\
1. \emph{John} must click on the button �VIRTUAL MACHINE� in the main Menu.\\
2. \emph{John} must click on the button with a plus named �CREATE VM�.\\
3. \emph{John} must click on the red button called 'PREDIFINED TEMPLATES'.\\ 
4. \emph{John} must enter on the top-left of the page a name for his virtual
machine and a description (optional).\\
5. \emph{John} must answer the different questions with yes or no.\\
6. \emph{John} must click on button called 'NEXT' and wait a couple of
seconds.\\
7. \emph{John} will see a preconfigured template that perfectly matches his
needs.\\
8. \emph{John} must click on the button called 'FINISH' to accept.


\item [\textbf{Extensions}]:\\
2.a John has to be logged in, in order to create a virtual machine.

\item [\textbf{GUI screenshot guide}]:\\
}
\end{lyxlist}
\hrule
\vspace{0.5cm}








\subsubsection{Modifying a virtual machine}

\hrule
\vspace{0.5cm}
\begin{lyxlist}{PC1}
\small{
\item [\textbf{Procedure:}] vmModificationProcess
\item [\textbf{Scope:}] Virtual machine modification
\item [\textbf{Primary Actor}:] SysAdmin John 
\item [\textbf{Secondary Actor(s)}:] /
\item [\textbf{Goal:}] The SysAdmin John should be able to modify a virtual
machine which was already created before.
\item [\textbf{Level}:] User-goal level
\item [\textbf{Main~Success~Scenario}]:\\
1. \emph{John} must click on the button �VIEW VM's� in the main Menu.\\
2. \emph{John} will see all the virtual machines that he created.\\
3. \emph{John} must select a virtual machine which he wants to modify by
clicking on it'.\\
4. \emph{John} must click now on the left button named `MODIFY`.\\
5. \emph{John} must now choose which components he wants to modify.
6. \emph{John} must click on the right black arrow to proceed to the next
component. 
7. \emph{John} must repeat step 5 for all the components that he
wants to change.\\
8. \emph{John} must now accept the modification(s) by clicking on the green
check mark.\\


\item [\textbf{Extensions}]:\\
2.a John has to be logged in, in order to create a virtual machine.\\
2.b John must have created at least one virtual machine before, in case to
perform a modification.\\

\item [\textbf{GUI screenshot guide}]:\\
}
\end{lyxlist}
\hrule
\vspace{0.5cm}









\subsubsection{Deleting a virtual machine}

\hrule
\vspace{0.5cm}
\begin{lyxlist}{PC1}
\small{
\item [\textbf{Procedure:}] vmDeletionProcess
\item [\textbf{Scope:}] Virtual machine deletion
\item [\textbf{Primary Actor}:] SysAdmin John 
\item [\textbf{Secondary Actor(s)}:] /
\item [\textbf{Goal:}] The SysAdmin John should be able to delete a virtual
machine which was already created before.
\item [\textbf{Level}:] User-goal level
\item [\textbf{Main~Success~Scenario}]:\\
1. \emph{John} must click on the button �VIEW VM's� in the main Menu.\\
2. \emph{John} will see all the virtual machines that he created.\\
3. \emph{John} must select a virtual machine which he wants to delete by
clicking on it'.\\
4. \emph{John} must click now on the middle button named `DELETE`.\\
5. \emph{John} must now click on the green check mark to delete the selected
VM.\\



\item [\textbf{Extensions}]:\\
2.a John has to be logged in, in order to create a virtual machine.\\
2.b John must have created at least one virtual machine inorder to perform a
deletion.\\

\item [\textbf{GUI screenshot guide}]:\\
}
\end{lyxlist}
\hrule
\vspace{0.5cm}







\subsubsection{Perform a hot backup}

\hrule
\vspace{0.5cm}
\begin{lyxlist}{PC1}
\small{
\item [\textbf{Procedure:}] vmBackupProcess1
\item [\textbf{Scope:}] Virtual machine backup
\item [\textbf{Primary Actor}:] SysAdmin John 
\item [\textbf{Secondary Actor(s)}:] /
\item [\textbf{Goal:}] The SysAdmin John should be able to perform a hot backup
on a selected virtual machine.
\item [\textbf{Level}:] User-goal level
\item [\textbf{Main~Success~Scenario}]:\\
1. \emph{John} must click on the button �VIEW VM'S� in the main Menu.\\
2. \emph{John} will see all the virtual machines that he created.\\
3. \emph{John} must select a virtual machine on which he wants to perform a
hot backup'.\\
4. \emph{John} must click now on the rightmost button named `BACKUP`.\\
5. \emph{John} must now click on the button named 'BACKUP NOW'.\\
6. \emph{John} must click on the green check mark to accept the requested
operation.\\



\item [\textbf{Extensions}]:\\
2.a John has to be logged in, in order to create a virtual machine.\\
2.b John must have created at least one virtual machine inorder to perform a
hot backup.\\

\item [\textbf{GUI screenshot guide}]:\\
}
\end{lyxlist}
\hrule
\vspace{0.5cm}










\subsubsection{Perform a scheduled backup}

\hrule
\vspace{0.5cm}
\begin{lyxlist}{PC1}
\small{
\item [\textbf{Procedure:}] vmBackupProcess2
\item [\textbf{Scope:}] Virtual machine backup
\item [\textbf{Primary Actor}:] SysAdmin John 
\item [\textbf{Secondary Actor(s)}:] /
\item [\textbf{Goal:}] The SysAdmin John should be able to perform a
scheduled backup on a selected virtual machine.
\item [\textbf{Level}:] User-goal level
\item [\textbf{Main~Success~Scenario}]:\\
1. \emph{John} must click on the button �VIEW VM's� in the main Menu.\\
2. \emph{John} will see all the virtual machines that he created.\\
3. \emph{John} must select a virtual machine on which he wants to perform a
hot backup'.\\
4. \emph{John} must click now on the rightmost button named `BACKUP`.\\
5. \emph{John} must now click on the button named 'SCHEDULED
BACKUP'.\\
6. \emph{John} can add a small description in the description label.\\
7. \emph{John} must specify a date by clicking on the calender on a specific
day.\\
8. \emph{John} must click on the button named 'PERFORM' to perform the backup.\\



\item [\textbf{Extensions}]:\\
2.a John has to be logged in, in order to create a virtual machine.\\
2.b John must have created at least one virtual machine inorder to perform a
cold/scheduled backup.\\

\item [\textbf{GUI screenshot guide}]:\\
}
\end{lyxlist}
\hrule
\vspace{0.5cm}















\subsection{super-SysAdmin}

\subsubsection{Overview of the datacenter}

\hrule
\vspace{0.5cm}
\begin{lyxlist}{PC1}
\small{
\item [\textbf{Procedure:}] DatacenterOverview
\item [\textbf{Scope:}] Datacenter Management
\item [\textbf{Primary Actor}:] super-SysAdmin Bob
\item [\textbf{Secondary Actor(s)}:] /
\item [\textbf{Goal:}] The super-SysAdmin Bob should be able to get an overview
of the datacenter
\item [\textbf{Level}:] User-goal level
\item [\textbf{Main~Success~Scenario}]:\\
1. \emph{Bob} has to click on the \emph{Hardware} button in the home screen.\\
2. \emph{Bob} has the choice to overview a component individually.\\
3. \emph{Bob} clicks on the component he would like to overview e.g. CPU, GPU,
RAM, HDD, SSD or Software\\
4. \emph{Bob} gets the overview of the component he chose poping up on the
rigt side of the screen.\\
5. \emph{Bob} now can see what has been used and what not.\\
6. \emph{Bob} after he's done with overview the choosen component \emph{Bob} can
go back to the super-SysAdmin home screen by clicking on the \emph{Back}
button.\\


\item [\textbf{Extensions}]:\\
2.a \emph{Bob} has to be an super-SysAdmin and has to be logged in in order to
be able to overview the datacenter's usage.\\
}
\end{lyxlist}
\hrule


\subsubsection{Buy new component(s)}

\hrule
\vspace{0.5cm}
\begin{lyxlist}{PC1}
\small{
\item [\textbf{Procedure:}] ComponentPurchase
\item [\textbf{Scope:}] Purchase of a/multiple component(s)
\item [\textbf{Primary Actor}:] super-SysAdmin Bob
\item [\textbf{Secondary Actor(s)}:] /
\item [\textbf{Goal:}] The super-SysAdmin Bob should be able to buy new
component(s)
\item [\textbf{Level}:] User-goal level
\item [\textbf{Main~Success~Scenario}]:\\
1. \emph{Bob} has to click on the \emph{Harware} button in the home screen.\\
2. \emph{Bob} chooses the component he wants to buy.\\
3. \emph{Bob} clicks on the component he would like to buy e.g. CPU, GPU,
RAM, HDD, SSD or Software\\
4. \emph{Bob} gets the overview of the component and clicks on the \emph{Buy
new InsertComponentName} button.\\
5. \emph{Bob} now chooses which material of a component he wants to buy by
selecting it and gives in the amount.\\
6. \emph{Bob} clicks on the \emph{Buy} button.\\
7. \emph{Bob} confirms the purchase.\\
8. \emph{Bob} clicks on the \emph{Back} button.\\


\item [\textbf{Extensions}]:\\
2.a \emph{Bob} has to be an super-SysAdmin and has to be logged in in order to
be able to buy a new component(s).\\
}
\end{lyxlist}
\hrule


\subsubsection{Creating a SysAdmin}

\hrule
\vspace{0.5cm}
\begin{lyxlist}{PC1}
\small{
\item [\textbf{Procedure:}] SysAdminCreation 
\item [\textbf{Scope:}] Allow the super administrator to create a new SysAdmin
\item [\textbf{Primary Actor}:] super-SysAdmin Bob
\item [\textbf{Secondary Actor(s)}:] /
\item [\textbf{Goal:}] The super-SysAdmin Bob should be able to create a new
regular user.
\item [\textbf{Level}:] User-goal level
\item [\textbf{Main~Success~Scenario}]:\\
1. \emph{Bob} has to click on the "CREATE NEW USER '' button in the super-SysAdmin home
screen.\\
2. \emph{Bob} has to fill the different textfields.\\
3. \emph After the different texfields have been filled. {Bob} has to click on
the "next step" button.\\
4. \emph{Bob} has to attribute the different rights for the SysAdmin.
{Bob} need only to click in the litle box if he want to attribute a right.\\
5. \emph{Bob} only needs now to click on the ``Create user" button for create a
SysAdmin.
\\
6. \emph A new SysAdmin has been created and {Bob} only needs to click
on the " ok" button.\\

\item [\textbf{Extensions}]:\\
2.a \emph{Bob} has to be an super-SysAdmin and has to be logged in in order to
be able to create a regular user.\\
}
\end{lyxlist}


\subsection{My-Actor2}

\subsubsection{MyProcedure1MyActor2}
\ldots

\subsubsection{MyProcedure2MyActor2}
\ldots














