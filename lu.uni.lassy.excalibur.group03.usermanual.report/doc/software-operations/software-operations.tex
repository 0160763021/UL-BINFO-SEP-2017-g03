\chapter{Software operations}
\label{chap:soptware_operations}


Explain each allowed software operations (i.e. an atomic unit of treatment, a service, a functionality) including a brief description of the operation, required parameters, optional parameters, default options, required steps to trigger the operation, assumptions upon request of the operation and expected results of executing such operation.
Describe how to recognise that the operation has successfully been executed or
abnormally terminated. The template given below (i.e. section \ref{operation:MyOperation} has to be used).

Group the operations devoted to the needs of specific actors. Common
operations to several actors may be grouped and presented once to avoid redundancy.


\section{createVMExpert}
\label{operation:createVMExpert}
This system operation allows to create a virtual machine for a regular user
using expert mode.

\begin{description}

\item \textbf{Parameters:} CPU, GPU, RAM, SSD, HDD, SOFTWARE
\item \textbf{Precondition:} Inorder to create a virtual machine the user must
be logged in.
\item \textbf{Post-condition:} A new virtual machine will be created and a new
request will be sent to an administrator.
\item \textbf{Output messages:} The operation will send at the end two
notification one for the administrator and one for the regular user. The first
message is used for the administrator so that he nows that there is a new
request for creating a new virtual machine. The second message will tell the
regular user that his request for the virtual machine was succesfully sent to an
administrator. 

\item \textbf{Triggering:}
\begin{enumerate}
\item Click on one of the radio buttons to choose one component.
\item Click on the right black arrow to proceed to the new component.
\item Repeat the two above steps until you arrive at the finish page.
\item Click on the green arrow to accept or the red cross to cancel the request.
\end{enumerate}

 
\end{description}

 
\subsection{createVMExpertExample}
Examples should illustrate the use of \textbf{complex operations}.

Each example must show how the actor uses the software operation under
description to achieve (at least one of) its expected outcome.

It might be required to include GUI screenshots to illustrate the example.







\section{createVMTemplate}
\label{operation:createVMTemplate}
This system operation allows to create a virtual machine for a regular user
using predifined template.

\begin{description}

\item \textbf{Parameters:} Template
\item \textbf{Precondition:} Inorder to create a virtual machine the user must
be logged in.
\item \textbf{Post-condition:} A new virtual machine will be created and a new
request will be sent to an administrator.
\item \textbf{Output messages:} The operation will send on notification to the
administrator telling which template the user desires to be used for his virtual
machine. The regular user will see a pop up message telling him that his request
was succesfully sent.

\item \textbf{Triggering:}
\begin{enumerate}
\item Click on one of the radio buttons to answer the question, the left one
represents 'yes' the right one 'no.
\item Repeat the first step until you have answer to every question.
\item Click on next button which is situated at the right corner of the page
called 'NEXT'.
\item Wait ten seconds until the system has proposed a template.
\item Click on the right button at the right corner of the page called 'FINISH'
to send the request.
\end{enumerate}

 
\end{description}

 
\subsection{createVMExpertExample}
Examples should illustrate the use of \textbf{complex operations}.

Each example must show how the actor uses the software operation under
description to achieve (at least one of) its expected outcome.

It might be required to include GUI screenshots to illustrate the example.



