\chapter{Software operations}
\label{chap:soptware_operations}


Explain each allowed software operations (i.e. an atomic unit of treatment, a service, a functionality) including a brief description of the operation, required parameters, optional parameters, default options, required steps to trigger the operation, assumptions upon request of the operation and expected results of executing such operation.
Describe how to recognise that the operation has successfully been executed or
abnormally terminated. The template given below (i.e. section \ref{operation:MyOperation} has to be used).

Group the operations devoted to the needs of specific actors. Common
operations to several actors may be grouped and presented once to avoid redundancy.


\section{createVMExpert}
\label{operation:createVMExpert}
This system operation allows a SysAdmin to create a virtual machine using expert
mode. Expert mode means that the SysAdmin can specify each component.

\begin{description}

\item \textbf{Parameters:} CPU, GPU, RAM, SSD, HDD, SOFTWARE
\item \textbf{Precondition:} Inorder to create a virtual machine the SysAdmin must
be logged in.
\item \textbf{Post-condition:} A new virtual machine will be created using
expert mode.
\item \textbf{Output messages:} The operation will send at the end one
notification telling the SysAdmin that he succefully created the desired virtual
machine.

\item \textbf{Triggering:}
\begin{enumerate}
\item Click on one of the radio buttons to choose one component.
\item Click on the right black arrow to proceed to the new component.
\item Repeat the two above steps until you arrive at the finish page.
\item Click on the green arrow to create the desired virtual machine.
\end{enumerate}

 
\end{description}

 
\subsection{createVMExpertExample}
Examples should illustrate the use of \textbf{complex operations}.

Each example must show how the actor uses the software operation under
description to achieve (at least one of) its expected outcome.

It might be required to include GUI screenshots to illustrate the example.









\section{createVMTemplateExample}
\label{operation:createVMTemplateExample}
This system operation allows a SysAdmin to create a virtual machine using
template mode. Template mode means the virtual machine is already predifined and
preconfigured.
\begin{description}

\item \textbf{Parameters:} Template
\item \textbf{Precondition:} Inorder to create a virtual machine the SysAdmin must
be logged in.
\item \textbf{Post-condition:} A new virtual machine will be created using
template mode.
\item \textbf{Output messages:} The operation will send at the end one
notification telling the SysAdmin that he succefully created the desired virtual
machine.

\item \textbf{Triggering:}
\begin{enumerate}
\item Click on one of the radio buttons to answer to the
question, the left radio button represents 'YES' the right one 'NO'.
\item Repeat the first step for all the questions.
\item Click on the button which is situated at the right corner of the page
called 'NEXT'.
\item Wait ten seconds until the system has proposed a template.
\item Click on the right button at the right corner of the page called 'FINISH'
to create the virtual machine.
\end{enumerate}

 
\end{description}

 
\subsection{createVMTemplateExample}
Examples should illustrate the use of \textbf{complex operations}.

Each example must show how the actor uses the software operation under
description to achieve (at least one of) its expected outcome.

It might be required to include GUI screenshots to illustrate the example.














\section{modifyVM}
\label{operation:modifyVM}
This system operation allows a SysAdmin to modify a virtual machine. 
\begin{description}

\item \textbf{Parameters:} CPU, GPU, RAM, SSD, HDD, SOFTWARE
\item \textbf{Precondition:} Inorder to modify a virtual machine the SysAdmin
must be logged in and he must have created at least one virtual machine.
\item \textbf{Post-condition:} The desired virtual machine will be modified
depending on the new components choosen by the SysAdmin
\item \textbf{Output messages:} The operation will send at the end one
notification telling the SysAdmin that he succefully modified the desired
virtual machine.

\item \textbf{Triggering:}
\begin{enumerate}
\item Click on one of the radio buttons to choose one component that you want
to modify.
\item Click on the right black arrow to proceed to the new component.
\item Repeat the two above steps for all components that you want to change
until you arrive at the finish page.
\item Click on the green check mark to accept the modification.
\end{enumerate}

 
\end{description}

 
\subsection{modifyVM}
Examples should illustrate the use of \textbf{complex operations}.

Each example must show how the actor uses the software operation under
description to achieve (at least one of) its expected outcome.

It might be required to include GUI screenshots to illustrate the example.












\section{deleteVM}
\label{operation:deleteVM}
This system operation allows a SysAdmin to delete a virtual machine. 
\begin{description}

\item \textbf{Parameters:} VirtualMachine
\item \textbf{Precondition:} Inorder to delete a virtual machine the SysAdmin
must be logged in and he must have created at least one virtual machine.
\item \textbf{Post-condition:} The desired virtual machine will be deleted.
\item \textbf{Output messages:} The operation will send at the end one
notification telling the SysAdmin that he succefully deleted the desired
virtual machine.

\item \textbf{Triggering:}
\begin{enumerate}
\item Click on the green check mark to accept the deletion.
\end{enumerate}

 
\end{description}

 
\subsection{deleteVM}
Examples should illustrate the use of \textbf{complex operations}.

Each example must show how the actor uses the software operation under
description to achieve (at least one of) its expected outcome.

It might be required to include GUI screenshots to illustrate the example.











\section{hotBackupVM}
\label{operation:hotBackupVM}
This system operation allows a SysAdmin to perform a hot backup on a virtual
machine.
\begin{description}

\item \textbf{Parameters:} VirtualMachine
\item \textbf{Precondition:} Inorder to perform a hot backup on a virtual
machine the SysAdmin must be logged in and he must have created at least one
virtual machine.
\item \textbf{Post-condition:} A hot backup will be performed on the selected
VM.
\item \textbf{Output messages:} The operation will send at the end one
notification telling the SysAdmin that he succefully performed a hot backup on
the desired virtual machine.

\item \textbf{Triggering:}
\begin{enumerate}
\item Click on the green check mark to accept the operation.
\end{enumerate}

 
\end{description}

 
\subsection{hotBackupVM}
Examples should illustrate the use of \textbf{complex operations}.

Each example must show how the actor uses the software operation under
description to achieve (at least one of) its expected outcome.

It might be required to include GUI screenshots to illustrate the example.








\section{scheduledBackupVM}
\label{operation:scheduledBackupVM}
This system operation allows a SysAdmin to perform a scheduled backup on a
virtual machine.
\begin{description}

\item \textbf{Parameters:} VirtualMachine
\item \textbf{Precondition:} Inorder to perform a scheduled backup on a virtual
machine the SysAdmin must be logged in and he must have created at least one
virtual machine.
\item \textbf{Post-condition:} A scheduled backup will be performed on the
selected VM.
\item \textbf{Output messages:} The operation will send at the end one
notification telling the SysAdmin that he succefully performed a scheduled
backup on the desired virtual machine.

\item \textbf{Triggering:}
\begin{enumerate}
\item Click on the description label and add a description. (optional)
\item Click on the calandar and select a day.
\item Click on the button named 'PERFORM'.
\end{enumerate}

 
\end{description}


\subsection{scheduledBackupVM}
Examples should illustrate the use of \textbf{complex operations}.

Each example must show how the actor uses the software operation under
description to achieve (at least one of) its expected outcome.

It might be required to include GUI screenshots to illustrate the example.






\section{DatacenterOverview}
\label{operation:datacenteroverview}
This system operation allows the super-SysAdmin to have an overview on the
datacenter.

\begin{description}

\item \textbf{Parameters:} CPU, GPU, RAM, SSD, HDD, SOFTWARE
\item \textbf{Precondition:} The person must be the super-SysAdmin and be logged
in with the super-SysAdmin's account.
\item \textbf{Post-condition:} The super-SysAdmin will have an idea of the
overall usage of each component by the virtual achines.
\item \textbf{Output messages:}

\item \textbf{Triggering:}
\begin{enumerate}
\item Click on one of the component's icon to get its overall usage.
\item Repeat the above step for each component the super-SysAdmin wants to have a
look at.
\end{enumerate}

 
\end{description}

 
\subsection{DatacenterOverviewExample}
Examples should illustrate the use of \textbf{complex operations}.

Each example must show how the actor uses the software operation under
description to achieve (at least one of) its expected outcome.

It might be required to include GUI screenshots to illustrate the example.






\section{ComponentRequest}
\label{operation:componentrequest}
This system operation allows the super-SysAdmin to buy request component(s) for
the datacenter.

\begin{description}

\item \textbf{Parameters:} CPU, GPU, RAM, SSD, HDD, SOFTWARE
\item \textbf{Precondition:} The person must be the super-SysAdmin and be logged
in with the super-SysAdmin's account.
\item \textbf{Post-condition:} The SysAdmin will have more components
available to create virtual machines.
\item \textbf{Output messages:} A request message will be send to have more
components available.

\item \textbf{Triggering:}
\begin{enumerate}
\item Click on the chosen radio button of each component.
\item Repeat the above step on the second page.
\item Click on the confirm icon to send a request.
\item A message confirming that a request has been send appears.
\end{enumerate}

 
\end{description}

\subsection{ComponentRequestExample}
Examples should illustrate the use of \textbf{complex operations}.

Each example must show how the actor uses the software operation under
description to achieve (at least one of) its expected outcome.

It might be required to include GUI screenshots to illustrate the example.




\section{createSysAdmin}
\label{operation:createSysAdmin}
This system operation allows the super-SysAdmin to create a SysAdmin.

\begin{description}

\item \textbf{Parameters:} Name, Username, E-mail, Password, Birth date, Phone
number, Rights
\item \textbf{Precondition:} The person must be the super-SysAdmin and be logged
in with the super-SysAdmin's account.
\item \textbf{Post-condition:} A new SysAdmin will be created and this
SysAdmin will be able to access the application.
\item \textbf{Output messages:} The operation will send one notification to the
super-SysAdmin telling that a SysAdmin user has been successfully created.


\item \textbf{Triggering:}
\begin{enumerate}
\item Click on the differents textfields an fill them with the appropriate
value.
\item Make sure that all different textfields has been filled before to
click on the ''Next step'' button which is situated at the right corner of the
page.
\item A new page ''Attribute the rights'' appears to give the SysAdmin the
power to attribute the new SysAdmin rights about the application.
\item Click on the right button at the right corner of the page called
''Create User'' to create the new SysAdmin.
\item A confirmation message will appears to make sure that a new SysAdmin
has been created. Click on the ''ok'' button to finish the operation.
\end{enumerate}

 
\end{description}

 
\subsection{createSysAdminExample}
Examples should illustrate the use of \textbf{complex operations}.

Each example must show how the actor uses the software operation under
description to achieve (at least one of) its expected outcome.

It might be required to include GUI screenshots to illustrate the example.








\section{Change E-mail}
\label{operation:changeEmail}
This system operation allows the super-SysAdmin or the SysAdmin to change
their E-mail.

\begin{description}

\item \textbf{Parameters:} newEmail
\item \textbf{Precondition:} The person must be the super-SysAdmin or the
SysAdmin and be logged in with the super-SysAdmin's account for the
super-SysAdmin or with the SysAdmin's account for the SysAdmin.
\item \textbf{Post-condition:}The email that is currently active will be
replaced by a new email that is given by the current user.
\item \textbf{Output messages:} The operation will send one notification to the
user telling that the E-mail has successfully been modified.


\item \textbf{Triggering:}
\begin{enumerate}
\item Click on the "Change Email " icon to begin with the changeEmail operation. 
\item A new page''Confirm the password'' appears. You need to insert your
account password in the input text field. 
\item Make sure that you insert the right password before to click on the
''confirm'' button otherwise you can not access the next step.
\item A new page ''Insert your new E-mail!'' appears and allows you to
insert a new e-mail in the input text field.
\item Click on the ''confirm'' button to replace the old e-mail by the new one.
\item A confirmation message will appears to make sure that the e-mail
has been changed. Click on the ''ok'' button to finish the operation.
\end{enumerate}

 
\end{description}

 
\subsection{changeEmailExample}
Examples should illustrate the use of \textbf{complex operations}.

Each example must show how the actor uses the software operation under
description to achieve (at least one of) its expected outcome.

It might be required to include GUI screenshots to illustrate the example.




\section{Change Id}
\label{operation:changeId}
This system operation allows the super-SysAdmin or the SysAdmin to change
their Id.

\begin{description}

\item \textbf{Parameters:} newId
\item \textbf{Precondition:} The person must be the super-SysAdmin or the
SysAdmin and be logged in with the super-SysAdmin's account for the
super-SysAdmin or with the SysAdmin's account for the SysAdmin.
\item \textbf{Post-condition:}The Id that is currently active will be
replaced by a new Id that is given by the current user.
\item \textbf{Output messages:} The operation will send one notification to the
current user telling that the Id  has successfully been modified.


\item \textbf{Triggering:}
\begin{enumerate}
\item Click on the "Change ID " icon to begin with the changeId operation. 
\item A new page''Confirm the password'' appears. You need to insert your
account password in the input text field. 
\item Make sure that you insert the right password before to click on the
''confirm'' button otherwise you can not access the next step.
\item A new page ''Insert your new Id'' appears and allows you to
insert a new Id in the input text field.
\item Click on the ''confirm'' button to replace the old Id by the new one.
\item A confirmation message will appears to make sure that the Id
has been changed. Click on the ''ok'' button to finish the operation.
\end{enumerate}

 
\end{description}

 
\subsection{changeIdExample}
Examples should illustrate the use of \textbf{complex operations}.

Each example must show how the actor uses the software operation under
description to achieve (at least one of) its expected outcome.

It might be required to include GUI screenshots to illustrate the example.



\section{Change Password}
\label{operation:changePassword}
This system operation allows the super-SysAdmin or the SysAdmin to change
their Password.

\begin{description}

\item \textbf{Parameters:} newPassword
\item \textbf{Precondition:} The person must be the super-SysAdmin or the
SysAdmin and be logged in with the super-SysAdmin's account for the
super-SysAdmin or with the SysAdmin's account for the SysAdmin.
\item \textbf{Post-condition:}The password that is currently active will be
replaced by a new password that is given by the current user.
\item \textbf{Output messages:} The operation will send one notification to the
current user telling that the password  has successfully been modified.


\item \textbf{Triggering:}
\begin{enumerate}
\item Click on the "Change Password " icon to begin with the changePassword
operation.
\item A new page''Confirm the password'' appears. You need to insert your old
account password in the input text field. 
\item Make sure that you insert the right password before to click on the
''confirm'' button otherwise you can not access the next step.
\item A new page ''Insert your new Password'' appears and allows you to
insert a new password in the input text field.
\item Click on the ''confirm'' button to replace the old password by the new
one.
\item A confirmation message will appears to make sure that the password
has been changed. Click on the ''ok'' button to finish the operation.
\end{enumerate}

 
\end{description}

 
\subsection{changePasswordExample}
Examples should illustrate the use of \textbf{complex operations}.

Each example must show how the actor uses the software operation under
description to achieve (at least one of) its expected outcome.

It might be required to include GUI screenshots to illustrate the example.




\section{Delete account}
\label{operation:deletAccount}
This system operation allows the SysAdmin to delete his account.

\begin{description}

\item \textbf{Parameters:} SysadminAccount
\item \textbf{Precondition:} The person must be a SysAdmin and be logged in with
his SysAdmin's account.
\item \textbf{Post-condition:}The account that is currently active and used by
the SysAdmin will be deleted and the SysAdmin will no longer have access to his
account.
\item \textbf{Output messages:} The operation will send one notification to the
current user telling that his account has been deleted.


\item \textbf{Triggering:}
\begin{enumerate}
\item Click on the "Delete Account " icon to begin with the DeletAccount
operation.
\item A new page''Confirm the password'' appears. You need to insert your
account password in the input text field. 
\item Make sure that you insert the right password before to click on the
''confirm'' button otherwise you can not access the next step.
\item A new confirmation page appears and ask you if you really want to delete
your account.
\item If you really want to delete your account you have to click on the
''Yes'' button. You have to take note that all your personal data will be
deleted and you will also no longer have access to your account.
\item If you click on the ''Yes'' button you account is delete and a
confirmation message will appears to make sure that the account has been
deleted. Click on the ''ok'' button to finish the operation.
\item If you click on the ''No'' button your account is not delete and the
operation will finish.
\end{enumerate}

 
\end{description}

 
\subsection{DeleteAccount}
Examples should illustrate the use of \textbf{complex operations}.

Each example must show how the actor uses the software operation under
description to achieve (at least one of) its expected outcome.

It might be required to include GUI screenshots to illustrate the example.
