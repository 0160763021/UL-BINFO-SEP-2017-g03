 \chapter{Software operations}
\label{chap:soptware_operations}


Explain each allowed software operations (i.e. an atomic unit of treatment, a service, a functionality) including a brief description of the operation, required parameters, optional parameters, default options, required steps to trigger the operation, assumptions upon request of the operation and expected results of executing such operation.
Describe how to recognise that the operation has successfully been executed or
abnormally terminated. The template given below (i.e. section \ref{operation:MyOperation} has to be used).

Group the operations devoted to the needs of specific actors. Common
operations to several actors may be grouped and presented once to avoid redundancy.



\section{CreateSystemAndEnvironment}
\label{operation:CreateSystemAndEnvironment}
This software operation initializes the system.
\begin{description}

\item \textbf{Parameters:} CPUName, CpuModel, CpuClockSpeed, CpuCores,
CpuAmount, GpuName, GpuModel, GpuClockSpeed, GpuCores, GpuAmount, RamName,
RamModel, RamCapacity, RamAmount, HDDName, HDDModel, HDDCapacity, HDDAmount,
SSDName, SSDModel, SSDCapacity, SSDAmount, SoftwareName, SoftwareModel,
SoftwareLicenseID, SoftwareAmount
\item \textbf{Precondition:} /
\item \textbf{Post-condition:} /
\item \textbf{Output messages:} /

\item \textbf{Triggering:}
\begin{enumerate}
\item 
\end{enumerate}

 
\end{description}

\subsection{CreateSystemAndEnvironment}
Examples should illustrate the use of \textbf{complex operations}.

Each example must show how the actor uses the software operation under
description to achieve (at least one of) its expected outcome.

It might be required to include GUI screenshots to illustrate the example.









\section{InitVM}
\label{operation:InitVM}
This software operation initializes a virtual machine with a name, desription
and a unique identifier.
\begin{description}

\item \textbf{Parameters:} VMID, VMName, VMDescription
\item \textbf{Precondition:} Inorder to create a virtual machine the SysAdmin must
specify a name and a description for the virtual machine. The sysAdmin has to be
logged in.
\item \textbf{Post-condition:} The sysAdmin will be allowed to proceed to the
next screen which allows him to choose the components of the virtual machine
that he wants to create.
\item \textbf{Output messages:} The operation will send a notification in form
of an error message in case the sysAdmin has not specified a name or/and a
description for the virtual machine.

\item \textbf{Triggering:}
\begin{enumerate}
\item Click on the button called 'NEXT'.
\end{enumerate}

 
\end{description}

\subsection{InitVM}
Examples should illustrate the use of \textbf{complex operations}.

Each example must show how the actor uses the software operation under
description to achieve (at least one of) its expected outcome.

It might be required to include GUI screenshots to illustrate the example.








\section{getAvailableComponents}
\label{operation:getAvailableComponents}
This software operation returns a list of all the components which are available
in the datacenter and can be used for the creation of a virtual machine.
\begin{description}

\item \textbf{Parameters:} etTypeOfComponent
\item \textbf{Precondition:} There must be at least one component available for
the requested type of component.
\item \textbf{Post-condition:} The software operation will return a list
representing all the available components for the given type which are available
in the datacenter and can later be used by the sysAdmin to create a virtual machine.

\item \textbf{Output messages:} The operation will send a notification in form
of an error message in case the datacenter lacks of ressources.

\item \textbf{Triggering:}
\begin{enumerate}
\item Click on the black arrow to proceed to the next or previous type of
component.
\end{enumerate}

 
\end{description}

\subsection{getAvailableComponents}
Examples should illustrate the use of \textbf{complex operations}.

Each example must show how the actor uses the software operation under
description to achieve (at least one of) its expected outcome.

It might be required to include GUI screenshots to illustrate the example.







\section{ValidateVM}
\label{operation:ValidateVM}
This software operation allows a SysAdmin to create a virtual machine, but
before doing that the sysAdmin must validate the creation.

\begin{description}

\item \textbf{Parameters:} VMID
\item \textbf{Precondition:} /
\item \textbf{Post-condition:} A new virtual machine will be created.
\item \textbf{Output messages:} The operation will send at the end one
notification telling the SysAdmin that he succefully created the desired virtual
machine.

\item \textbf{Triggering:}
\begin{enumerate}
\item Click on the green arrow to create the desired virtual machine.
\end{enumerate}

 
\end{description}

 
\subsection{ValidateVM}
Examples should illustrate the use of \textbf{complex operations}.

Each example must show how the actor uses the software operation under
description to achieve (at least one of) its expected outcome.

It might be required to include GUI screenshots to illustrate the example.








\section{GenerateComponent}
\label{operation:GenerateComponent}
This software operation will generate a component depending on the answer of the
question.
\begin{description}

\item \textbf{Parameters:} VMID, QuestionID, etAnswer
\item \textbf{Precondition:} /
\item \textbf{Post-condition:} The software operation will return a component
which then can be used to create a virtual machine
\item \textbf{Output messages:} /

\item \textbf{Triggering:}
\begin{enumerate}
\item Click on the left radio button to answer the question with ’YES’ or on 
the right radio button to answer the question with ’NO’

\end{enumerate}

 
\end{description}

\subsection{GenerateComponent}
Examples should illustrate the use of \textbf{complex operations}.

Each example must show how the actor uses the software operation under
description to achieve (at least one of) its expected outcome.

It might be required to include GUI screenshots to illustrate the example.









\section{GetProposedVM}
\label{operation:GetProposedVM}
This software operation generates a virtual machine and will have as components
the components which were generated before, depending on the answers of the
different questions.
\begin{description}

\item \textbf{Parameters:} VMID
\item \textbf{Precondition:} /
\item \textbf{Post-condition:} The software operation will return a virtual
machine that was generated by the system.
\item \textbf{Output messages:} The operation will send at the end one
notification informing the sysAdmin that the virtual machine was successfully
generated.

\item \textbf{Triggering:}
\begin{enumerate}
\item Click on the button called 'NEXT'
\end{enumerate}

 
\end{description}

\subsection{GetProposedVM}
Examples should illustrate the use of \textbf{complex operations}.

Each example must show how the actor uses the software operation under
description to achieve (at least one of) its expected outcome.

It might be required to include GUI screenshots to illustrate the example.

























\section{SetComponent}
\label{operation:SetComponent}
This software operation specifies a component for a virtual machine
\begin{description}

\item \textbf{Parameters:} VMID, ComponentID, ComponentAmount,
etTypeOfComponent
\item \textbf{Precondition:} Inorder to set a component there must be a least
one component available for the specified type of component.
\item \textbf{Post-condition:} The virtual machine will be set with a new
component of the specified type.
\item \textbf{Output messages:} /

\item \textbf{Triggering:}
\begin{enumerate}
\item Click on one of the radio buttons to choose one component.
\item Click on the black arrow to proceed to the next or previous type of
component.
\end{enumerate}

 
\end{description}

 
\subsection{SetComponent}
Examples should illustrate the use of \textbf{complex operations}.

Each example must show how the actor uses the software operation under
description to achieve (at least one of) its expected outcome.

It might be required to include GUI screenshots to illustrate the example.












\section{DeleteVM}
\label{operation:DeleteVM}
This software operation allows a SysAdmin to delete a virtual machine. 
\begin{description}

\item \textbf{Parameters:} VMID, sysAdminPassword
\item \textbf{Precondition:} Inorder to delete a virtual machine the sysAdmin
needs to be logged in and provide the correct password for his account.

\item \textbf{Post-condition:} The desired virtual machine will be deleted.
\item \textbf{Output messages:} The operation will send at the end one
notification telling the SysAdmin that he succefully deleted the desired
virtual machine or it will send one notification in form of an error message in
case the sysAdmin has not provided his correct password.

\item \textbf{Triggering:}
\begin{enumerate}
\item Click on the green check mark to accept the deletion.
\end{enumerate}

 
\end{description}

 
\subsection{DeleteVM}
Examples should illustrate the use of \textbf{complex operations}.

Each example must show how the actor uses the software operation under
description to achieve (at least one of) its expected outcome.

It might be required to include GUI screenshots to illustrate the example.











\section{HotBackupVM}
\label{operation:HotBackupVM}
This software operation allows a SysAdmin to perform a hot backup on a virtual
machine.
\begin{description}

\item \textbf{Parameters:} VMIdentifier, currentDate
\item \textbf{Precondition:} Inorder to perform a hot backup on a virtual
machine the SysAdmin must be logged in.
\item \textbf{Post-condition:} A hot backup will be performed on the selected
VM.
\item \textbf{Output messages:} The operation will send at the end one
notification telling the SysAdmin that he succefully performed a hot backup on
the desired virtual machine.

\item \textbf{Triggering:}
\begin{enumerate}
\item Click on the green check mark to accept the operation.
\end{enumerate}

 
\end{description}

 
\subsection{HotBackupVM}
Examples should illustrate the use of \textbf{complex operations}.

Each example must show how the actor uses the software operation under
description to achieve (at least one of) its expected outcome.

It might be required to include GUI screenshots to illustrate the example.







\section{ScheduledBackupVM}
\label{operation:ScheduledBackupVM}
This software operation allows a SysAdmin to perform a scheduled backup on a
virtual machine.
\begin{description}

\item \textbf{Parameters:} VMID, BackupDescription, Date
\item \textbf{Precondition:} Inorder to perform a scheduled backup the sysAdmin 
must select a date and also a backup description.
\item \textbf{Post-condition:} A scheduled backup will be performed on the
selected VM.
\item \textbf{Output messages:} The software operation will send a message in
form of an error message, in case the date is missing or if the sysAdmin has 
not specified a description for the backup. If all the necessary information
is provided a message will be sent informing the SysAdmin that the scheduled 
backup was succesfully created


\item \textbf{Triggering:}
\begin{enumerate}
\item Click on the calandar and select a day.
\item Click on the button named 'PERFORM'.
\end{enumerate}

 
\end{description}


\subsection{ScheduledBackupVM}
Examples should illustrate the use of \textbf{complex operations}.

Each example must show how the actor uses the software operation under
description to achieve (at least one of) its expected outcome.

It might be required to include GUI screenshots to illustrate the example.



\section{Delete account}
\label{operation:ConfirmDelete}
This system operation allows the SysAdmin to delete his account.

\begin{description}

\item \textbf{Parameters:} SysadminAccount
\item \textbf{Precondition:} The person must be a SysAdmin or a super-sysAdmin
and be logged in with his SysAdmin's account. Before to be able to delete the
account, the person has to confirm his password.
\item \textbf{Post-condition:}The account that is currently active and used by
the SysAdmin will be deleted and the SysAdmin will no longer have access to his
account.
\item \textbf{Output messages:} The operation will send one notification to the
current user telling that his account has been deleted.


\item \textbf{Triggering:}
\begin{enumerate}
\item Click on the "Delete Account " icon to begin with the DeletAccount
operation.
\item A new page''Confirm the password'' appears. You need to insert your
account password in the input text field. 
\item Make sure that you insert the right password before to click on the
''confirm'' button otherwise you can not access the next step.
\item A new confirmation page appears and ask you if you really want to delete
your account.
\item If you really want to delete your account you have to click on the
''Yes'' button. You have to take note that all your personal data will be
deleted and you will also no longer have access to your account.
\item If you click on the ''Yes'' button you account is delete and a
confirmation message will appears to make sure that the account has been
deleted. Click on the ''ok'' button to finish the operation.
\item If you click on the ''No'' button your account is not delete and the
operation will finish.
\end{enumerate}

 
\end{description}

 
\subsection{DeleteAccount}
Examples should illustrate the use of \textbf{complex operations}.

Each example must show how the actor uses the software operation under
description to achieve (at least one of) its expected outcome.

It might be required to include GUI screenshots to illustrate the example.



\section{OverviewComponent}
\label{operation:overviewcomponent}
This system operation allows the super-SysAdmin to have an overview of the
datacenter.

\begin{description}

\item \textbf{Parameters:} etComponent
\item \textbf{Precondition:} The person must be the super-SysAdmin and be logged
in with the super-SysAdmin's account.
\item \textbf{Post-condition:} The super-SysAdmin will know the usage of each
component he chose to overview.
\item \textbf{Output messages:}

\item \textbf{Triggering:}
\begin{enumerate}
\item Click on one of the component's icon to get its overview.
\item Repeat the above step for each component the superSysAdmin want to
overview.
\end{enumerate}

 
\end{description}

\subsection{OverviewComponentExample}
Examples should illustrate the use of \textbf{complex operations}.

Each example must show how the actor uses the software operation under
description to achieve (at least one of) its expected outcome.

It might be required to include GUI screenshots to illustrate the example.




\section{ChooseModels}
\label{operation:choosemodels}
This system operation allows the super-SysAdmin to choose between the different
models of the component he want to request to the datacenter.

\begin{description}

\item \textbf{Parameters:} ComponentID, Amount
\item \textbf{Precondition:} The person must be the super-SysAdmin and be logged
in with the super-SysAdmin's account.
\item \textbf{Post-condition:} The SysAdmin will have more components/models
available to create virtual machines.
\item \textbf{Output messages:}

\item \textbf{Triggering:}
\begin{enumerate}
\item Click on the button of the component model you want to request.
\item Click on the quantity select list next to it to give the amount to
request.
\item Repeat the above step for each model you want.
\item Click on the \emph{Next} button to go to the confirmation page.
\end{enumerate}

 
\end{description}

\subsection{ChooseModelsExample}
Examples should illustrate the use of \textbf{complex operations}.

Each example must show how the actor uses the software operation under
description to achieve (at least one of) its expected outcome.

It might be required to include GUI screenshots to illustrate the example.







\section{ConfirmChoice}
\label{operation:confirmchoice}
This system operation allows the super-SysAdmin to confirm the component models
he chose before to send a request to the datacenter.

\begin{description}

\item \textbf{Parameters:} /
\item \textbf{Precondition:} The person must be the super-SysAdmin and be logged
in with the super-SysAdmin's account.
\item \textbf{Post-condition:} The SysAdmin will send a request to the
datacenter for more components.
\item \textbf{Output messages:} A request has been successfully send to the
datacenter receiver.

\item \textbf{Triggering:}
\begin{enumerate}
\item Click the yes button to confirm the request.
\item When you click on the X button you abort the request.
\end{enumerate}

 
\end{description}

\subsection{ConfirmChoiceExample}
Examples should illustrate the use of \textbf{complex operations}.

Each example must show how the actor uses the software operation under
description to achieve (at least one of) its expected outcome.

It might be required to include GUI screenshots to illustrate the example.












\section{PersonalInformation}
\label{operation:PersonalInformation}
This system operation allows the super-SysAdmin to create a SysAdmin.

\begin{description}

\item \textbf{Parameters:} FirsName,LastName, Username, E-mail, Password, Birth
date, Phone number
\item \textbf{Precondition:} The person must be the super-SysAdmin and be logged
in with the super-SysAdmin's account.
\item \textbf{Post-condition:} Peronal information of a sysadmin are saved
for the cration of an sys-admin
\item \textbf{Output messages:} 


\item \textbf{Triggering:}
\begin{enumerate}
\item Click on the differents textfields an fill them with the appropriate
value.
\item Click on the ''Next step'' button which is situated at the right corner of
the page call the operation.


\end{enumerate}

 
\end{description}

 
\subsection{PersonalInformation}
Examples should illustrate the use of \textbf{complex operations}.

Each example must show how the actor uses the software operation under
description to achieve (at least one of) its expected outcome.

It might be required to include GUI screenshots to illustrate the example.






\section{AttributeRights}
\label{operation:AttributeRights}
This system operation allows the super-sysAdmin to attribute rights.

\begin{description}

\item \textbf{Parameters:} UserName,Right1, Right2, Right3, Right4, Right5,
Right6, Right7, Right8
\item \textbf{Precondition:} The person must be the super-SysAdmin and be logged
in. Before to have acces to the AttributeRights operation the super-sysadmin
hass to complete the PersonalInformation operation.
\item \textbf{Post-condition:} Rights will be attributed to an sysAdmin that is
being created.
\item \textbf{Output messages:} The operation will send one notification to the
super-SysAdmin telling that a SysAdmin user has been successfully created.


\item \textbf{Triggering:}
\begin{enumerate}
 \item After having completed the PersonalInformation operation a new page
 ''Attribute the rights'' appears to give the Super-SysAdmin the power to attribute the new SysAdmin rights about the application.
\item The super-sysadmin can assign a right by clicking on a box.
\item Click on the right button at the right corner of the page called
''Create User'' to call the "Attribute Rights" operation.

\end{enumerate}

 
\end{description}

 
\subsection{AttributeRights}
Examples should illustrate the use of \textbf{complex operations}.

Each example must show how the actor uses the software operation under
description to achieve (at least one of) its expected outcome.

It might be required to include GUI screenshots to illustrate the example.






\section{ConfirmPassword}
\label{operation:ConfirmPassword}
This system operation confirms if the password given by an actor is the right
one. It compares the given password with the right password.

\begin{description}

\item \textbf{Parameters:} inputPassword
\item \textbf{Precondition:} The person must be the super-SysAdmin or the
SysAdmin and be logged in.
\item \textbf{Post-condition:}A comparation is made between the actor's given
password and the right password.
\item \textbf{Output messages:} 


\item \textbf{Triggering:}
\begin{enumerate}
 \item The actor need to fill the password textfield with the appropriate
 password .
\item Click on the ''Confirm'' to call the operation that compares the given
pasword with the right password.

\end{enumerate}

 
\end{description}

 
\subsection{ConfirmPassword}
Examples should illustrate the use of \textbf{complex operations}.

Each example must show how the actor uses the software operation under
description to achieve (at least one of) its expected outcome.

It might be required to include GUI screenshots to illustrate the example.







\section{Change E-mail}
\label{operation:NewEmail}
This system operation allows the super-SysAdmin or the SysAdmin to change
their E-mail.

\begin{description}

\item \textbf{Parameters:} newEmail
\item \textbf{Precondition:} The person must be the super-SysAdmin or the
SysAdmin and be logged in. Before to be able to change the e-mail, the person
has to confirm his password.
\item \textbf{Post-condition:}The email that is currently active will be
replaced by a new email that is given by the current user.
\item \textbf{Output messages:} The operation will send one notification to the
user telling that the E-mail has successfully been modified.


\item \textbf{Triggering:}
\begin{enumerate}
\item Click on the "Change Email " icon to begin with the changeEmail operation. 
\item A new page''Confirm the password'' appears. You need to insert your
account password in the input text field. 
\item Click on the''confirm'' button otherwise you can not access the next
step.
\item A new page ''Insert your new E-mail!'' appears and allows you to
insert a new e-mail in the input text field.
\item Click on the ''confirm'' button to replace the old e-mail by the new one.

\end{enumerate}

 
\end{description}

 
\subsection{changeEmailExample}
Examples should illustrate the use of \textbf{complex operations}.

Each example must show how the actor uses the software operation under
description to achieve (at least one of) its expected outcome.

It might be required to include GUI screenshots to illustrate the example.




\section{Change Id}
\label{operation:NewId}
This system operation allows the super-SysAdmin or the SysAdmin to change
their Id.

\begin{description}

\item \textbf{Parameters:} newId
\item \textbf{Precondition:} The person must be the super-SysAdmin or the
SysAdmin and be logged in. Before to be able to change the id, the person has to
confirm his password.
\item \textbf{Post-condition:}The Id that is currently active will be
replaced by a new Id that is given by the current user.
\item \textbf{Output messages:} The operation will send one notification to the
current user telling that the Id  has successfully been modified.


\item \textbf{Triggering:}
\begin{enumerate}
\item Click on the "Change ID " icon to begin with the changeId operation. 
\item A new page''Confirm the password'' appears. You need to insert your
account password in the input text field. 
\item Click on the ''confirm'' button otherwise you can not access the next
step.
\item A new page ''Insert your new Id'' appears and allows you to
insert a new Id in the input text field.
\item Click on the ''confirm'' button to replace the old Id by the new one.

\end{enumerate}

 
\end{description}

 
\subsection{changeIdExample}
Examples should illustrate the use of \textbf{complex operations}.

Each example must show how the actor uses the software operation under
description to achieve (at least one of) its expected outcome.

It might be required to include GUI screenshots to illustrate the example.



\section{Change Password}
\label{operation:NewPw}
This system operation allows the super-SysAdmin or the SysAdmin to change
their Password.

\begin{description}

\item \textbf{Parameters:} newPassword
\item \textbf{Precondition:} The person must be the super-SysAdmin or the
SysAdmin and be logged in. Before to be able to change the password, the person has to confirm his password.
\item \textbf{Post-condition:}The password that is currently active will be
replaced by a new password that is given by the current user.
\item \textbf{Output messages:} The operation will send one notification to the
current user telling that the password  has successfully been modified.


\item \textbf{Triggering:}
\begin{enumerate}
\item Click on the "Change Password " icon to begin with the changePassword
operation.
\item A new page''Confirm the password'' appears. You need to insert your old
account password in the input text field. 
\item Click on the ''confirm'' button otherwise you can not access the next
step.
\item A new page ''Insert your new Password'' appears and allows you to
insert a new password in the input text field.
\item Click on the ''confirm'' button to replace the old password by the new
one.

\end{enumerate}

 
\end{description}

 
\subsection{changePasswordExample}
Examples should illustrate the use of \textbf{complex operations}.

Each example must show how the actor uses the software operation under
description to achieve (at least one of) its expected outcome.

It might be required to include GUI screenshots to illustrate the example.

