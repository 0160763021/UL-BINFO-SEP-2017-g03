\chapter{Software operations}
\label{chap:soptware_operations}


Explain each allowed software operations (i.e. an atomic unit of treatment, a service, a functionality) including a brief description of the operation, required parameters, optional parameters, default options, required steps to trigger the operation, assumptions upon request of the operation and expected results of executing such operation.
Describe how to recognise that the operation has successfully been executed or
abnormally terminated. The template given below (i.e. section \ref{operation:MyOperation} has to be used).

Group the operations devoted to the needs of specific actors. Common
operations to several actors may be grouped and presented once to avoid redundancy.


\section{createVMExpert}
\label{operation:createVMExpert}
This system operation allow a SysAdmin to create a virtual machine using expert
mode.

\begin{description}

\item \textbf{Parameters:} CPU, GPU, RAM, SSD, HDD, SOFTWARE
\item \textbf{Precondition:} Inorder to create a virtual machine the SysAdmin must
be logged in.
\item \textbf{Post-condition:} A new virtual machine will be created using
expert mode.
\item \textbf{Output messages:} The operation will send at the end one
notification telling the SysAdmin that he succefully created the desired virtual
machine.

\item \textbf{Triggering:}
\begin{enumerate}
\item Click on one of the radio buttons to choose one component.
\item Click on the right black arrow to proceed to the new component.
\item Repeat the two above steps until you arrive at the finish page.
\item Click on the green arrow to accept or the red cross to cancel the request.
\end{enumerate}

 
\end{description}

 
\subsection{createVMExpertExample}
Examples should illustrate the use of \textbf{complex operations}.

Each example must show how the actor uses the software operation under
description to achieve (at least one of) its expected outcome.

It might be required to include GUI screenshots to illustrate the example.







\section{createVMTemplate}
\label{operation:createVMTemplate}
This system operation allow a SysAdmin to create a virtual machine using
template mode.
\begin{description}

\item \textbf{Parameters:} Template
\item \textbf{Precondition:} Inorder to create a virtual machine the SysAdmin must
be logged in.
\item \textbf{Post-condition:} A new virtual machine will be created using
template mode.
\item \textbf{Output messages:} The operation will send at the end one
notification telling the SysAdmin that he succefully created the desired virtual
machine.

\item \textbf{Triggering:}
\begin{enumerate}
\item Click on one of the radio buttons to answer the question, the left one
represents 'yes' the right one 'no.
\item Repeat the first step until you have answer to every question.
\item Click on next button which is situated at the right corner of the page
called 'NEXT'.
\item Wait ten seconds until the system has proposed a template.
\item Click on the right button at the right corner of the page called 'FINISH'
to send the request.
\end{enumerate}

 
\end{description}

 
\subsection{createVMTemplateExample}
Examples should illustrate the use of \textbf{complex operations}.

Each example must show how the actor uses the software operation under
description to achieve (at least one of) its expected outcome.

It might be required to include GUI screenshots to illustrate the example.





\section{DatacenterOverview}
\label{operation:datacenteroverview}
This system operation allows the super-SysAdmin to have an overview on the
datacenter.

\begin{description}

\item \textbf{Parameters:} CPU, GPU, RAM, SSD, HDD, SOFTWARE
\item \textbf{Precondition:} The person must be the super-SysAdmin and be logged
in with the super-SysAdmin's account.
\item \textbf{Post-condition:} The super-SysAdmin will have an idea of the
overall usage of each component by the virtual achines.
\item \textbf{Output messages:}

\item \textbf{Triggering:}
\begin{enumerate}
\item Click on one of the components to get its overall usage.
\item Repeat the above step for each component the super-SysAdmin wants to have a
look at.
\end{enumerate}

 
\end{description}

 
\subsection{DatacenterOverviewExample}
Examples should illustrate the use of \textbf{complex operations}.

Each example must show how the actor uses the software operation under
description to achieve (at least one of) its expected outcome.

It might be required to include GUI screenshots to illustrate the example.






\section{ComponentPurchase}
\label{operation:componentpurchase}
This system operation allows the super-SysAdmin to buy new component(s) for the
datacenter.

\begin{description}

\item \textbf{Parameters:} CPU, GPU, RAM, SSD, HDD, SOFTWARE
\item \textbf{Precondition:} The person must be the super-SysAdmin and be logged
in with the super-SysAdmin's account.
\item \textbf{Post-condition:} The SysAdmin will have more components
available to create virtual machines.
\item \textbf{Output messages:} A request will be send to the datacenter in
order to install the new purchased components physically.

\item \textbf{Triggering:}
\begin{enumerate}
\item Check the box of the material to buy and specify the amount.
\item Click on the \emph{Buy} button.
\item Confirm the purchase.
\item A message confirming that a request has been send to the datacenter
appears. Click on the \emph{Back} button to finish the operation.
\end{enumerate}

 
\end{description}

\subsection{ComponentPurchaseExample}
Examples should illustrate the use of \textbf{complex operations}.

Each example must show how the actor uses the software operation under
description to achieve (at least one of) its expected outcome.

It might be required to include GUI screenshots to illustrate the example.






\section{createSysAdmin}
\label{operation:createSysAdmin}
This system operation allows the super-SysAdmin to create a SysAdmin.

\begin{description}

\item \textbf{Parameters:} Name, Username, E-mail, Password, Birth date, Phone
number, Rights
\item \textbf{Precondition:} The person must be the super-SysAdmin and be logged
in with the super-SysAdmin's account.
\item \textbf{Post-condition:} A new SysAdmin will be created and this
SysAdmin will be able to access the application.
\item \textbf{Output messages:} The operation will send one notification to the
super-SysAdmin telling that a SysAdmin user has been successfully created.


\item \textbf{Triggering:}
\begin{enumerate}
\item Click on the differents textfields an fill them with the appropriate
value.
\item Make sure that all different textfields has been filled before to
click on the ''Next step'' button which is situated at the right corner of the
page.
\item A new page ''Attribute the rights'' appears to give the SysAdmin the
power to attribute the new SysAdmin rights about the application.
\item Click on the right button at the right corner of the page called
''Create User'' to create the new SysAdmin.
\item A confirmation message will appears to make sure that a new SysAdmin
has been created. Click on the ''ok'' button to finish the operation.
\end{enumerate}

 
\end{description}

 
\subsection{createRegularUserExample}
Examples should illustrate the use of \textbf{complex operations}.

Each example must show how the actor uses the software operation under
description to achieve (at least one of) its expected outcome.

It might be required to include GUI screenshots to illustrate the example.
