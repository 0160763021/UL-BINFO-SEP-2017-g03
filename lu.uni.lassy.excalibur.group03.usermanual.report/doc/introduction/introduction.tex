\chapter{Introduction}
\label{chap:introduction}

\section{Scope}
 
This section has to provide the scope of the user's manual document.
In the following some opening statements to use when providing the
information corresponding to this section.

This document provides all essential information for the user to make full
use of the software \mysystemname
%Example: This document provides minimum acceptable information for knowing how
% to use the software system \mysystemname.


This document does not include any technical support that is not directly
related with our software.\ldots
 
This document is not intented to provid information about how the application
internally works, configure any additional hardware or other technical
support. Third-party software system that is rqeuired for the correct
funcitoning of \mysystemname.

 
This document may be used with other documents provided by third-party
companies to have an overall view and correct understanding of the environment
and procedures where the software system \mysystemname is aimed to be deployed
and run.




\section{Purpose}
In this section you explain the purpose (i.e. aim, objectives) of the user's
manual. In the following some examples of opening statements to be used in this
section.

The purpose of this document is to provide all necessary information
needed to use the software VMMS and benefit from all the advantages that the
application provides.

This document defines \ldots

This document is meant to \ldots



\section{Intended audience}
Description of the categories of persons targeted by this document together with the description of how they are expected to exploit the content of the document.
The intended audience of our software a
This document is target at two categories of persons, both of them are human
actors and are the followed: SysAdmin's and super-SysAdmin.\\\\


\section{\mysystemname}

VMMS is a simple web-oriented application aimed for individuals with the
objectiv to create and manage their own virtual machine. The application    
has two perspectives, one when logged in as SysAdmin and another when
logged in as an super-SysAdmin. As an super-SysAdmin you get the option to look
at the ressources utilization of the datacenter as well as the deployed virtual
machines by the SysAdmin's while SysAdmin has the option to create or modify 
a VM using his own preferences or by choosing a predined template.\\The benefits
for the client is to save on hardware without sacrificing on reliability and 
flexibility. At any point in time the SysAdmin can connect to the web-oriented
application where he gets access to all the created virtual machines and
personal account information about himself.\\For a super-SysAdmin using our
application means getting easy access and flexibility in terms of displaying and managing the deployed VMs.

\subsection{Actors \& Functionalities}




\subsection{Operating environment}
Brief overview of the infrastructure on which the software is deployed and used.

\section{Document structure}  
The user manual is divided into the following sections and is meant to be read
in the ordering as written here. All the sections are directed to both the sysadmin
and the super-SysAdmin beside the chapter 3 and 4, since both of them will
have different fonctionalities how they interact with the system:\\\\

Chapter 1 Contains information about the product, to be more precise about
VMMS and is subdivided into 8 different section which are the followed:

\begin{description}
\item[$\bullet$] Identification which should give the reader a short overview
of the user manual, as well as some important remarks that should be considered
by the user before using the software.
\item[$\bullet$] Copyright which informs the reader about the different rights
that apply  to the software and which are bound by the law of a country.
\item[$\bullet$] Trademark notices which identifies the product which is
provided  to the client.
\item[$\bullet$] Restrictions contains all the restrictions which apply to the
software when using it.
\item[$\bullet$] Warranties informs the reader about how to proceed in case of
software defects or failure.
\item[$\bullet$] Contractual obligations contain all the obligation which a
user is bound when using the software.
\item[$\bullet$] Disclaimers containing the responsibility of the user as well
as of VMMS.
\item[$\bullet$] Contact information can be found in this section.\\\\
\end{description}

Chapter 2 introduces the user manual and is subdivided into 5 different section
which are:


\begin{description}
\item[$\bullet$] Scope which provides the scope of the user manual.
\item[$\bullet$] Purpose contains the purpose of the user�s manual.
\item[$\bullet$] Intended audience contains all the information related to the 
different categories of persons to which VMMS is target to.
\item[$\bullet$] Detailed description about VMMS (Actors and Operating
environment).
\item[$\bullet$] Document structure\\\\
\end{description}


Chapter 3 provides information on how to use VMMS. This is done by describing
how each actor interacts with the system. The different steps of a procedure are
described in textual form as well as using images, showing step by step how to
arrive to the finality.\\\\

Chapter 4 provides information to all the supported operations of the system, by
specify to each operation which parameter must by specified as well as, a
step-by-step  guide, how to perform the operation.\\\\

Chapter 5 Here the reader can find all the relevant information related to
errors. Each error is described always using the same pattern, first identifying
the problem, and then providing a solution to the problem.



