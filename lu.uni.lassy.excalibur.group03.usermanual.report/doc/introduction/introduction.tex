\chapter{Introduction}
\label{chap:introduction}

\section{Scope}
 
This section has to provide the scope of the user's manual document.
In the following some opening statements to use when providing the
information corresponding to this section.

This document provides all essential information for the user to make full
use of the software \mysystemname
%Example: This document provides minimum acceptable information for knowing how
% to use the software system \mysystemname.


This document does not include any technical support that is not directly
related with our software.\ldots
 
This document is not intented to provid information about how the application
internally works, configure any additional hardware or other technical
support. Third-party software system that is rqeuired for the correct
funcitoning of \mysystemname.

 
This document may be used with other documents provided by third-party
companies to have an overall view and correct understanding of the environment
and procedures where the software system \mysystemname is aimed to be deployed
and run.




\section{Purpose}
In this section you explain the purpose (i.e. aim, objectives) of the user's
manual. In the following some examples of opening statements to be used in this
section.

The purpose of this document is to provide all necessary information
needed to use the software VMMS and benefit from all the advantages that the
application provides.

This document defines \ldots

This document is meant to \ldots



\section{Intended audience}
Description of the categories of persons targeted by this document together with the description of how they are expected to exploit the content of the document.
The intended audience of our software a
This document is target at two categories of persons, both of them are human
actors and are the followed: Regular users and Administrator.\\\\


\section{\mysystemname}

VMMS is a simple web-oriented application aimed for individuals with the
objectiv to create and manage their own virtual machine. The application    
has two perspectives, one when logged in as administrator and another when
logged in as an end-users. As an administrator you get the option to look at
the ressources utilization of the datacenter as well as the deployed virtual
machines by the end-users while the end user has the option to create or modify 
a VM using his own preferences or by choosing a predined template.\\The benefits
for the end-user is to save on hardware without sacrificing on reliability and 
flexibility. At any point in time the user can connect to the web-oriented
application where he gets access to his virtual machines and personal
account.\\For an administrator using our application means getting easy access
and flexibility in terms of displaying and managing the deployed VMs.

\subsection{Actors \& Functionalities}
Basic actors or regular user have limited rights compared to an administrator
and so both will have a different view of the software.\\\\The regular user will
have the possibility to:\\

\begin{description}
  \item[$\bullet$] Create an account
  \item[$\bullet$] Log in
  \item[$\bullet$] Make a request for creating a VM
  \item[$\bullet$] Make a request for deleting a VM
  \item[$\bullet$] Make a request for modifying a VM
  \item[$\bullet$] Send a message to an administrator
  \item[$\bullet$] View all his created virtual machines
  \item[$\bullet$] Create a snapshot of his virtual machine
  \item[$\bullet$] View his profile and change some settings. This includes all
  informations that the user had to provided when creating an account exept for
  the username which will not be possible to change
  \item[$\bullet$] Send a request to delete his account
  \item[$\bullet$] Log out
\end{description}



\subsection{Operating environment}
Brief overview of the infrastructure on which the software is deployed and used.

\section{Document structure}  
Information on how this document is organised and it is expected to be
used. Recommendations on which members of the audience
should consult which sections of the document, and explanations about the used
notation (i.e. description of formats and conventions) must also be provided.





