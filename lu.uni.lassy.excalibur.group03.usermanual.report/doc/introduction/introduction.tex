\chapter{Introduction}
\label{chap:introduction}

\section{Scope}
This document provides all the necessary information needed to correcly use the
software system \mysystemname and to benefit from all the fonctionalaties that
it provides.

This document is not intend to solve problems which are not directly directed to
the software \mysystemname itself. This includes problems related with hardware
issues, incompability issues or issues caused by third party services which are
needed to fully benefit from all the fonctionalities that \mysystemname
provides. It also not intended give an overview how the software \mysystemname
internally works neither to provide any technical support which is not directly 
addressed with the software \mysystemname.

This document is intended to be used with other documents provided by third
party companies to get a complet overview of the system \mysystemname and to
correct understand the environment and procedures where the software system
\mysystemname is aimed to be deployed and run.


\section{Purpose}
The purpose of this document is to provide all necessary information
needed to use the software VMMS and benefit from all the advantages that the
application provides.


\section{Intended audience}
This document is target to two different types of human actors wich are the 
followed: SysAdmin and super-SysAdmin.

The sysAdmin actor who is responsible for managing the different created an
deployed virutal machines in a datacenter.

The super-SysAdmin who is responsible for managing the ressources of the
datacenter. This ressources include material ressources but also
human-ressoucres such as adding new sysAdmin's to the system. All the sections
are intented to be used for both of the actors and is ment to provide all the 
necessary informations to correctly use the software.



\section{\mysystemname}

VMMS is a simple web-oriented application aimed for individuals with the
objectiv to create and manage their own virtual machine. The application    
has two perspectives, one when logged in as SysAdmin and another when
logged in as an super-SysAdmin. As an super-SysAdmin you get the option to look
at the ressources utilization of the datacenter as well as the deployed virtual
machines by the SysAdmin's while SysAdmin has the option to create or modify 
a VM using his own preferences or by choosing a predined template.\\The benefits
for the client is to save on hardware without sacrificing on reliability and 
flexibility. At any point in time the SysAdmin can connect to the web-oriented
application where he gets access to all the created virtual machines and
personal account information about himself.\\For a super-SysAdmin using our
application means getting easy access and flexibility in terms of displaying and
managing the deployed VMs.

\subsection{Actors \& Functionalities}



\textbf{Authenticated} Is an abstract actor which is inteded to make sure that
the different actors who log into the system can securely use the application. 
The functionalities are:

\begin{description}

\item[$\bullet$] Logging into the system
\item[$\bullet$] Logging out
\end{description}
\textbf{sysAdmin} The sysAdmin is responsible for creating, managing and
maintaining the deployed virtual machines. The fonctionalities for the sysAdmin 
are the followed:

\begin{description}
\item[$\bullet$] create a virtual mahine, either by choosing each component
individual or by using a preconfigured template
\item[$\bullet$] perform a backup, this backup can be a hot backup or a
scheduled one, so by specifying a date, when it should be performed
\item[$\bullet$] modify a virtual machine by selecting one or more new
components
\item[$\bullet$] delete a virtual machine
\item[$\bullet$] view its profile include all his personal information
\item[$\bullet$] change its personal information
\item[$\bullet$] delete his account
\end{description}
\textbf{superSysAdmin} The superSysAdmin is responsible for managing the
different sysAdmin's as well as the ressources of the datacenter. The
fonctionalities for the super-SysAdmin are the followed:
\begin{description}
\item[$\bullet$] request one or more components which then gets allocated to the
datacenter
\item[$\bullet$] create new sysAdmins and also to specify their access rights
\item[$\bullet$] get an overview of the ressources available in the datacenter
\end{description}

\textbf{Creator} Creators are responsible for installing and making sure that
the VMMS system is correctly deployed and fonctioning in its intended
environment.




\subsection{Operating environment}
The VMMS software system is a web-oriented application which can be accessed
from within a browser. The software itself runs on dedicated servers, on which also
the created virtual machines will run. The environment itself is flexible and so
it can be adapted depending on the situation and also on the needs of the datacenter.

\section{Document structure}  
The user manual is divided into the following sections and is meant to be read
in the ordering as written here. All the sections are directed to both the sysadmin
and the super-SysAdmin beside the chapter 3 and 4, since both of them will
have different fonctionalities how they interact with the system:\\\\

\textbf{Chapter 1} Contains information about the product, to be more precise
about VMMS and is subdivided into 8 different section which are the followed:

\begin{description}
\item[$\bullet$] Identification which should give the reader a short overview
of the user manual, as well as some important remarks that should be considered
by the user before using the software.
\item[$\bullet$] Copyright which informs the reader about the different rights
that apply  to the software and which are bound by the law of a country.
\item[$\bullet$] Trademark notices which identifies the product which is
provided  to the client.
\item[$\bullet$] Restrictions contains all the restrictions which apply to the
software when using it.
\item[$\bullet$] Warranties informs the reader about how to proceed in case of
software defects or failure.
\item[$\bullet$] Contractual obligations contain all the obligation which a
user is bound when using the software.
\item[$\bullet$] Disclaimers containing the responsibility of the user as well
as of VMMS.
\item[$\bullet$] Contact information can be found in this section.\\\\
\end{description}

\textbf{Chapter 2} introduces the user manual and is subdivided into 5 different
section which are:


\begin{description}
\item[$\bullet$] Scope which provides the scope of the user manual.
\item[$\bullet$] Purpose contains the purpose of the user manual.
\item[$\bullet$] Intended audience contains all the information related to the 
different categories of persons to which VMMS is target to.
\item[$\bullet$] Detailed description about VMMS (Actors and Operating
environment).
\item[$\bullet$] Document structure\\\\
\end{description}


\textbf{Chapter 3} provides information on how to use VMMS. This is done by
describing how each actor interacts with the system. The different steps of a procedure are
described in textual form as well as using images, showing step by step how to
arrive to the finality.\\\\

\textbf{Chapter 4} provides information to all the supported operations of the
system, by specify to each operation which parameter must by specified as well as, a
step-by-step  guide, how to perform the operation.\\\\

\textbf{Chapter 5} Here the reader can find all the relevant information related
to errors. Each error is described always using the same pattern, first identifying
the problem, and then providing a solution to the problem.



