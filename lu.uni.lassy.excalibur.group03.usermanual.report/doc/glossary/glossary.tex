\newglossaryentry{Glossary} 
{name={Glossary},
description={the description of terms that are
likely unfamiliar to the audience. The glossary shall include an alphabetical list of
terms and definitions. Documentation using abbreviations and acronyms unfamiliar
to the audience shall include a list with definitions, which may be integrated
with the glossary. Terms included in the glossary should also be defined on
their first appearance in printed documentation. Here there is an example of how
to include an expression into the glossary: \gls{socialware}. 
}, 
plural={Glossaries}
}



\newglossaryentry{Concept Model}
{name={Concept Model},
description={the Model that describes the different types required to specify
the software system.}, 
plural={Concept Models},
symbol={\msrglsstyle{Concept Model}}
}

\newglossaryentry{Environment Model} 
{name={Environment Model},
description={the Model that describes the different actors supposed to interact
with the software system.}, 
plural={Environment Models},
symbol={\msrglsstyle{Environment Model}}
}

\newglossaryentry{MVC} 
{name={Model-View-Controller},
description={the pattern followed to design the Graphical User Interfaces
of the software system.}, 
plural={Model-View-Controllers}, 
symbol={\msrglsstyle{Model-View-Controller}}
}


\newglossaryentry{Design Model} 
{name={Design Model},
description={The Design Class Model is composed of the contents of all design classes, i.e.
their (value) attributes and methods, all the navigable associations between
design classes, and the inheritance structure. The Design Class Model has to be
modelled as a UML Class Diagram.}, 
plural={Design Models}, 
symbol={\msrglsstyle{Design Model}}
}


\newglossaryentry{Interaction Model} 
{name={Interaction Model},
description={The Interaction Model shows how objects are expected to interact at run-time to
support the \emph{system operations} specified in the \emph{Operation Model}
made during the Analysis Phase. There must exist an \emph{Interaction Model} for
each system operation specified in the \emph{Operation Model}. An Interaction Model has to be
modelled as a UML Sequence Diagram.}, 
plural={Interaction Models}, 
symbol={\msrglsstyle{Interaction Model}}
}

\newglossaryentry{Deployment View} 
{name={Deployment View},
description={The physical view depicts the system from a system engineer's
point-of-view. It is concerned with the topology of software components on the
physical layer, as well as the physical connections between these components.
For example, how many nodes are used and what is deployed on what node. A
Deployment View is modelled as a UML Deployment Diagram.}, 
plural={Deployment Views}, 
symbol={\msrglsstyle{Deployment View}}
}


\newglossaryentry{Implementation View} 
{name={Implementation View},
description={This view describes the software system components. It focuses on
software modules and subsystems. It describes the hierarchies or layers for
components. This view is modelled as a UML Component Diagram.},
plural={Implementation Views}, 
symbol={\msrglsstyle{Implementation View}}
}


\newglossaryentry{Processing View} 
{name={Processing View},
description={This view deals with the dynamic aspects of the system. It is
aimed at describing processes taking part at runtime and, in particular, how
such process communicate among each other. A Processing View is modelled as a
UML Sequence Diagram.},
plural={Processing Views}, 
symbol={\msrglsstyle{Processing View}}
}


\newglossaryentry{Use-Case Instance View} 
{name={Use-Case Instance View},
description={The description of an architecture is illustrated using a small set
of use-case instances (or scenarios). The scenarios describe sequences of
interactions between objects, and between processes. They are used to identify
architectural elements and to illustrate and validate the architecture design.
They also serve as a starting point for tests of an architecture prototype. A
scenario has to be modelled as a UML Sequence Diagram.}, 
plural={Use-Case Instance Views}, 
symbol={\msrglsstyle{Use-Case Instance View}}
}

%  General Messir Glossary
\newglossaryentry{real number}
{name={Real number},
description={name of the set of real numbers.},
plural={reals},
symbol={\ensuremath{\mathbb{R}}}
}

\newglossaryentry{system operation}
{name={System Operation},
description={a functionality of the system that can be triggered by a message
sent by an actor belonging to the environment.}, plural={system operations},
symbol={\msrglsstyle{system operation}}
}


\newglossaryentry{societics}
{name={Societics},
description={Represents the fields of hardware/software
systems used for the society extension.}, 
symbol={\msrglsstyle{societics}}
}

\newglossaryentry{direct actor}
{name={Direct Actor},
description={an actor that interacts directly with the system. It thus belongs
to the environment.},
plural={direct actors},
symbol={\msrglsstyle{direct actor}}
}

\newglossaryentry{indirect actor}
{name={Indirect Actor},
description={an actor that interacts indirectly with the system through a direct
actor.  It thus belongs the domain but not to the environment.}, 
plural={indirect actors},
symbol={\msrglsstyle{indirect actor}}
}

\newglossaryentry{abstract actor}
{name={Abstract Actor},
description={an actor that does not exist in real life.},
plural={abstract actors},
symbol={\msrglsstyle{abstract actor}}
}

\newglossaryentry{socext}
{name={Society extension},
description={The society obtained by grouping people using natural means
extended with artificial means.},
symbol={\msrglsstyle{societics}}
}

\newglossaryentry{usecase}
{name={Use case},
description={A use case describes a sequence of actions that provide something
of measurable value to an actor. and is drawn as a horizontal ellipse.},
plural={Use cases}, 
symbol={\msrglsstyle{Use case}} 
}

\newglossaryentry{actor}
{name={Actor},
description={An actor is a person, organization, or external system that plays a
role in one or more interactions with the system.},
plural={actors},
symbol={\msrglsstyle{actor}}
}

\newglossaryentry{socialware}
{name={Societics},
description={Represents the fields of hardware/software
systems used for the society extension.},
symbol={\msrglsstyle{Societics}}
}

% \newglossaryentry{}
% {name={\msrglsstyle{}},
% description={},
% symbol={\msrglsstyle{actor}} }
